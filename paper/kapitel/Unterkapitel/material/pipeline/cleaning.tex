\paragraph{Data Cleaning}
\label{par:cleaning}
The quality of data can be described in many dimensions like \zb{} the trustworthiness of the data source, the consistency of the data,
or their accuracy as well as their topicality.
Just with very unstructured data, as in our case the collected dishes from the web page \url{speisekarte.de}, the cleaning of the data is
inevitable to get a good and accurate analysis result.
However, even with supposedly good data sources such as \acp{API}, there may be errors in the data.
such as \zb{} the mixing of different languages or the lack of information.
Differences in data quality can also occur in the process of data collection.

This paragraph explains the measures taken in this project to clean up the data collected.

\subparagraph{review\_count cleaning}
In the \pg{} database the review\_count of some restaurants was assigned the value \code{NULL}.
If a new comparison with the Yelp \ac{API} did not result in a change of the review count,
this \code{NULL} value was replaced by the number 0.
\subparagraph{address cleaning}
The \ac{API} of Yelp is designed so that an HTTP code 403 or 404 will be transmitted if no address to a restaurant
can be found.
With a separate python script the Yelp \ac{API} was addressed again to find the restaurant's address.
If still no city or postcode was found for this restaurant, this restaurant was deleted for reasons of simplicity,
since a restaurant without a postcode or city has no use for the subsequent analysis.
\subparagraph{price\_range cleaning}
Also in the price\_range attribute there were occasional occuring \code{NULL} values which possibly could be filled when addressing the Yelp \ac{API} a second time.
If this was not the case, the \code{NULL} values were filled with the mode of the price\_range from the current city.
\subparagraph{city cleaning}
Despite the fact that the \code{city} attribute of a restaurant always came from Yelp it came to a mixture of the German and English language.
The following city names had to be changed to their German counterparts:
\begin{enumerate}
  \item Hanover
  \item Dusseldorf
  \item Nuremberg
  \item Munich
\end{enumerate}
Furthermore, the German city \"Frankfurt am Main\" was available in different spellings.
However, this could be fixed with a simple \ac{SQL} statement.
\subparagraph{buying\_power cleaning}
The data source found for acquiring the buying\_power for various german cities unfortunately contained data for some of the german cities,
but the average buying\_power for germany.
For this reason, the missing buying\_power was replaced by the average buying\_power in Germany in order to be able to carry out at least a rough analysis.
\subparagraph{rent\_avg cleaning}
Also with the average purchase price of a restaurant per m\textsuperscript{2} the original data source did not include all cities of Germany and therefore
some \code{NULL} values remained in the database.
These have been replaced with the Germany-wide average.
\newline
For reasons of traceability, all values that were recalculated have been
not stored in the original table or overwritten the original values
but it was decided to store them in a separate table.
