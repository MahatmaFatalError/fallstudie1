\subsubsection{Data Sources}
\label{subsubsec:sources}
For the analysis of suitable locations for a restaurant, many different sources were found and integrated. The community website Yelp was one of the main data sources for the evaluations. The Yelp Fusion interface provides helpful information about restaurants and their quality.  Possible information is e.g. the name of the restaurant, the type of food, the number of ratings, the rating or the coordinates of the restaurant with latitude and longitude. However, there are limitations to the retention time of Yelp data \cite{yelp}, so a suitable infrastructure for data extraction and analysis should be available!
\newline
Buying Power is an important factor in the catering industry \cite{locana}. It makes little sense to open a three-star restaurant in a social focus area or vice versa. A fast food business will not last long between high-end haute cuisine restaurants. In 2016, public television broadcast a study on the quality of life in Germany. These studies included different population, city, and quality reports, with reference to external sources. One of the studies contained information on purchasing power in the cities of Germany. The report led to an external provider who had published this survey \cite{buyingpower}. This evaluation was publicly available in the download portal of the website in portable document format and was very well suited to adapting this with a python converter in the postgres database. This attribute complements the model to determine the price level for a restaurant.
\newline
For our use case and in general, it is absolutely necessary to know how much rent you can pay in the month or year or whether you can afford a property. In search of a rent index in Germany, we first came across an open-source evaluation in .json format \cite{Sparda}. Unfortunately, this was only the average land prices. Thereupon the possibility opened by an interface to real estate platforms \cite{ImmoScout} to get more accurate and more up-to-date rental and land prices.
\begin{itemize}
\item CSV file with the population and the area
\end{itemize}
