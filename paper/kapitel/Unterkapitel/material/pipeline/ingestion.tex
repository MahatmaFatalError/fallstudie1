\subsubsection{Data Ingestion}
\label{subsubsec:ingestion}
As descriped in\cite{ingestion} \textit{Data Ingestion} is the process of obtaining and importing data for immediate use or storage in a database.
In our usecase we are collecting data from various data sources, thus we need different implementation approaches on each kind of datasource.
\newline
Across all \code{collectors} data were collected from a total of five different types of data sources:
\begin{itemize}
  \item \ac{JSON}
  \item \ac{API}
  \item \ac{PDF}
  \item \ac{CSV}
  \item Web
\end{itemize}
A part that should not be underestimated is the resulting quality of the data.
While \ac{API}, \ac{CSV} and \ac{JSON} files consist of very structured data and reading them does not require much effort,
data from a \ac{PDF} file or homepage are unstructured and require a higher implementation effort and possibly an additional
\textit{Data Cleaning} step is required before using them in the analysis part to get the results that are truly reflecting the reality.
% For the implementation of \textit{Data Ingestion} the \code{collector.py} module was developed.
% Within this module is the base class for the flow logic of the concrete \textit{collectors}.
% Due to the inhomogeneity of the data sources and the associated different logic per \textit{collector}, the \code{Collector} class only contains the logic
% to store a newly created entity in the \gds{} and other auxiliary methods.
% That means that it is not possible to implement the flow logic in the base class, but be implemented to a large extent in each subclasses.
%
% To realize the \textbf{collector} and its subclasses the Python library \code{google-cloud-datastore} is used.
% The library \code{google-cloud-datastore} is a client with which it is possible to perform all known \ac{CRUD} operations for the \gds{}.
