\subsubsection{Data Storage}
\label{subsubsec:storage}
In order for the data to be available for analysis and visualization, it must be read from the \gds{} and imported into the \pg{} database.
Before the data in the \pg{} can be personalized, it must be mapped to the respective table format.
\newline
This process is called \textit{Data Storage} and the Python module which contains the whole Logic is called \code{transporter.py}.
If it is required to implement a new transporter the only part that has to be written is the mapping from a \gds{} Entity to a \pg{} table.
\newline
In summary, the following steps had to be taken to develop a new \textit{transporter}:
\begin{itemize}
  \item Create a new \pg{} table
  \item Create a new \code{<example>\_transporter.py} module that inherits from the base \code{Transporter} class.
  \item Implementation of the mapping method.
\end{itemize}
