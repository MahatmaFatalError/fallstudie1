\subsubsection{Data Storage}
\label{subsubsec:storage}
In order for the data to be available for analysis and visualization, it must be read from the \gds{} and imported into the \pg{} database.
Before the data in the \pg{} can be personalized, it must be mapped to the respective table format.

This process is called \textit{Data Storage} and the Python module \code{transporter.py} was developed especially for this purpose.
In contrast to the implementation of the \code{collector} module, a large part of the \textit{Data Storage} logic is in the base class.
Only the logic for transforming the \gds{} entity to a \pg{} database table must be implemented as soon as a new
data source is to be connected.

In summary, the following steps had to be taken to develop a new \textit{transporter}:
\begin{itemize}
  \item Create a new \pg{} table with \code{sqlalchemy}
  \item Create a new \code{<example>\_transporter.py} file that inherits from the base class.
  \item Implementation of the \code{map()} Methdode in which the \gds{} entity is received and one or more \code{sqlalchemy} object(s) are created and returned.
\end{itemize}
