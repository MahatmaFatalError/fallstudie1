\subsection{Review Analysis}
\label{subsec:review}
As mentioned in \fullref{sec:introduction} it is not only the goal to propose one or more cities for the opening of a restaurant,
but also to give recommendations for qualitative characteristics of a restaurant.
\newline
For this purpose, the available reviews of a city are analyzed with the help of \ac{NLP}.
\ac{NLP} is the computer-based attempt to support any text with the help of a set of technologies and linguistic knowledge.\cite{Liddy01}
The goal of \ac{NLP} is to achieve a human-like processing of language.\cite{Liddy01}
The frequency of a certain word within the reviews - and thus also the importance of the word - can be expressed by \ac{TFIDF}.
This \ac{KPI} consists of a total of two calculations.
First the \ac{TF} is calculated.
This indicates how often a single word appears within a given document.
Mathematically this means:
\newline
\begin{equation}
  tf\textsubscript{i,j} = \frac{n\textsubscript{i,j}}{\sum_{k} n\textsubscript{i,j}}
	\label{eq:tf}
\end{equation}
Since in the pure term-frequency the totality of documents is disregarded, there is the \ac{IDF} measure.
The \ac{IDF} measure indicates how often a single word occurs within the corpus.\footnote{The totality of all documents}
It is given by the equation below.
\newline
\begin{equation}
  idf(w) = \log \frac{N}{df\textsubscript{t}}
	\label{eq:tf}
\end{equation}
Where df\textsubscript{t} is the number of documents containing the term t and N is the total number of documents in the corpus.
The factor of these two figures then forms the \ac{TFIDF} measure.\cite{droid18}
\newline
For our purposes not every single word of an yelp review is of importance.
To filter words with a low information content from an existing text, there are several methods from the area of \textit{Text Preprocessing}.
The following methods were used to improve the information density of restaurant reviews in order to calculate a meaningful \ac{TFIDF}.
\paragraph{Tokenization}
Tokenization is the process to divide a coherent sentence into single pieces, called tokens.
Beside the subdivision into single pieces, the so called stopwords and the punctuation is removed.
After Tokenization is executed on this example sentence:
\newline
\textitbf{\code{Yes, the 5 stars are deserved: here you can drink and buy the best coffee in Bochum (and maybe in the Ruhr area?).}}
\newline
only the following list of words/tokens will remain:
\newline
\textitbf{\code{['yes', '5', 'stars', 'deserved', 'drink', 'buy', 'best', 'coffee', 'bochum', 'maybe', 'ruhr', 'area']}}
\newline
The sentence structure and the grammatical correctness of the sentence are lost, but the words with a high expressiveness remain.
\paragraph{Stemming}
In linguistic morphology and information retrieval, stemming is the process for reducing inflected (or sometimes derived) words to their stem.\cite{TextMiner14}
As a result, the number of words to be analyzed is reduced and the frequency of a word is increased by reducing it to the trunk.
The following figure illustrates this fact:
\bild{stemming.png}{Stemming example}{stemming}{own representation}{0.75}
\paragraph{\acs{POS} Tagging}
\ac{POS} Tagging is another important topic in the area of \ac{NLP}.
With Tagging, words in a document are getting marked with a tag.
This tag describes the individual components of a document with their respective types \eg{} nouns, adjectives, verbs, separators, numbers, currencies and so on.
With this it is possible to filter out the types that are insignificant for the analysis, such as numbers or currencies, thus further reducing the number of words and
to increase the meaning of the remaining words.
\newline
By evaluating the most frequently used words, values can be found out on which characteristics the restaurant visitors attach the most importance.
In addition, grouping ratings into very good (5 stars) and very bad (1 star) gives the possibility to give recommendations for different restaurant characteristics which will increase the likelihood of very good ratings
and minimizes the risk of very poor ratings.
\newline
\fullref{app:review} shows words with the highest \ac{TFIDF}, which are occurring in 5-star reviews in Bochum.
