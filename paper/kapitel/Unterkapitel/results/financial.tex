\subsection{Investment Analysis}
\label{subsec:moneten}
The target, mentioned in \fullref{sec:introduction}, is to generate a revenue of 40,000 \euro{} per month. For this purpose
 a budget of 750,000 \euro{} is available. Based on the analyzed Top 10 cities, all affordable real estate objects will be
 selected. Therefor the following steps are necessary:
 \newline
 \begin{itemize}
   \item Calculate the required total floor space and the number of seats for the restaurant
   \item Looking for real estate objects which match these requirements
   \item Filter only affordable real estate objects
 \end{itemize}
\subsubsection{Calculate the required total floor space and the number of seats for a restaurant}
In order to generate this above-mentioned monthly revenue a calculation with parameters such as revenue per seat and day,
 opening days, average consumption per guest and seat and the area of the guest section are required. To calculate the
 minimum number of seats the restaurant needs to reach the monthly revenue is:
\begin{equation}
\begin{aligned}
	min.number.seats & = \lceil\frac{revenue.restaurant.month}{revenue.seat.day \times avg.consumption.guest.day \times opening.days.month}\rceil \\
	& = \lceil\frac{40,000 \ \text{\euro{}} }{1.5 \times 20 \ \text{\euro{}} \times 26}\rceil \\
	& = 52
\end{aligned}
\label{eq:number_seats_benchmark}
\end{equation}
where $revenue.seat.day.$ and $avg.consumption.guest.day$ given by \cite{BenchmarkGastronomie}. $opening.days.month$
 are calculated as usual in the gastronomy as 6 days a week. For monthly analysis the 6 days per week are multiplied by
\begin{equation}
    \begin{aligned}
        opening.days.month & = week.working.days \times avg.weeks.year \\
        & = 6 \ days \times \frac{52 \ weeks.year}{12 \ month.year} \\
        & = 26 \ days
    \end{aligned}
    \label{eq:opening_days}
\end{equation}
wihtin a month.\newline
With having the rounded minimum number of seats the computation of the total floor space is next. The total
 floor space parameter is necessary for the specific query of the ImmobilienScout24 Search API \cite{ImmoScout} and
 the number of seats as well. The number of seats can be applied as range of values. The lower threshold is calculated to 52 seats and
 the upper threshold is 65 seats. The upper threshold is defined as an increase of 25 percent of the minimum number of seats.
 This interval can be used to calculate the required total floor area using the following formula:
\begin{equation}
    \begin{aligned}
        total.floor.area.min = floor.area.guest \times seats.min \times \frac{100}{40} \\
        total.floor.area.max = floor.area.guest \times seats.max \times \frac{100}{40} \\
    \end{aligned}
    \label{eq:total_floor_space}
\end{equation}
where $floor.area.guest$ is the average floor space for a System-Service-Restaurant per guest is 1.5 square meter \cite{FlaecheGast} and the
 amount of the guest area is 40 percent mentioned in \cite{FlaecheGastronomie}.
\subsubsection{Looking for real estate objects that matching these requirements}
In order to apply the financing analysis, the Geo-ID of the top 10 cities is queried using the GeoAutoCompletion (GAC)
 v2 \ac{API} of ImmobilienScout24 \cite{ImmoScout}. Having this specific ID the real estate object can be queried via ImmobilienScout24 Search \ac{API}
 with the defined parameters of the Region Search Query. The used Parameters are $min.number.seats$,
 $max.number.seats$, $total.floor.area.min$, $total.floor.area.max$, the real estate specific parameter
 \code{gastronomy} and the gastronomy-type \code{restaurant}. The results are stored in the \pg{} database.
\subsubsection{Filter only affordable real estate objects}
The found real estate objects will be checked for affordability. The created equation is
    \begin{equation}
        \begin{aligned}
            rest.budget & = available.budget \\
                & - price.real.estate.object \times multiplier \\
                & - furnish.costs.sqm \times total.floor.space \\
                & - seats.min \times furnish.costs.seat
        \end{aligned}
        \label{eq:affordable}
    \end{equation}
where multiplier is depending on whether the object is payed onetime only or by rent every month. If the real estate
 object has a rental property the multiplier is set to twelve otherwise it's one. The $furnish.costs.sqm$ and
 $furnish.costs.seat$ are parameters used from \cite{BenchmarkGastronomie}. $price.real.estate.object$ and
 $total.floor.space$ where given through the result of the ImmobilienScout24 Search \ac{API}.\newline
 This calculation will be applied to
 every object and if the \textit{rest.budget} is greater or equal zero the object is affordable. For some results the
 \textit{rest.budget} is zero because there are some real estate objects on ImmobilienScout24 where their
 price is only available on request, so their price is presented as zero but their are also interesting.
\newline
Finally, it should be mentioned that the results are volatile and depending on the current real estate market.
