\section{Literature Synthesis}
\label{sec:synthesis}
There are many papers present which are referring to the restaurant business but most of them either have a different point of view or are focusing on a very specific part like the siting of a restaurant or the impact of online ratings.

The latter aspect is deeply discussed in the paper from Havard Unversity\cite{CaseOfYelp}. In this the author elaborates on the significant influence of Yelp user ratings on the revenue of restaurants. His investigation is based on data of Washington’s State Department of Revenue. Hence, he proves the positive effect of customer ratings on a restaurant's revenue, at least for non-chain affiliated restaurants in the United States. This is definitely a valuable insight which acts as a baseline of the analysis part for this paper since there is no equivalent accounting data of independent restaurants in Germany available. 

Second, the bachelor thesis of Hasan\cite{Imatra} is about market analysis of existing restaurants based on various research methods. His goal is decision support on what kind of restaurant should be opened in a given city in eastern Finland. His approach heavily relies on surveys of the local population. Hence, he focuses on a standardized questionnaires rather than the broad variety of online reviews. Moreover, in contrast to the problem statement of this paper he does not aim to find a optimal location as he restricted his research to a narrow region. 

In the paper \cite{SentimentAnalysis} the authors proposed a method to identify the sentiment tendency of restaurant reviews and to classify them to a specific type of restaurant. Although their analysis is based on the Yelp Dataset Challenge\footnote{\href{https://www.yelp.com/dataset/challenge}{www.yelp.com/dataset/challenge}} and thus seems to be closely related to this paper, they neither provide insights for non-english reviews nor a holistic investigation that exceeds online reviews which is a major goal of this paper.
