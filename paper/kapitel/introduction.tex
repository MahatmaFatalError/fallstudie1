\section{Introduction}
\label{sec:introduction}
%Allgemeine Einführung in das Thema
Nowadays lots of people want to realize their dream to start an own business - often its associated with a own restaurant. Therefor a well-conceived concept is absolutely necessary because you have to deal with a lot of different requirements in relation to the financial aspects, the siting and the customers. All these factors should be well-matched to have a unique feature as a restaurant and to be competitive. \newline
%Überleitung zu den Lösungsansätzen
The purpose of the thesis is to make a recommendation for a new restaurant business in Germany. Especially to find a good city for opening a restaurant based on different facts as well as reaching the mentioned financial specifications of the investment.
A good indicator to discover a new restaurant business is to analyze restaurants which already exist. In the meantime the world wide web and it's community platforms are inevitable. Customers can leave reviews and rate their restaurant visits online. One of this platforms with it's thousands of restaurants and other business cases is definitely Yelp. Yelp offers an API for developers which can be accessed to collect the restaurant data, e.g. the name, the rating, the city and so on. For a significant analysis more there was more information needed. As a consequence we collected more relevant key figures which help to find an appropriate city, especially with the dataset of all cities in Germany and its population as well as the datasets which include the buying power and the rent average for industrial buildings. The main idea was to combine all the mentioned key figures to create a own \ac{KPI} from it to find out the most appropriate cities  in Germany for starting a restaurant business. Furthermore we analyzed the user's text review's on yelp to get more insights about the existing restaurants and their favorite dishes of different restaurant categories. As to that the menu's of an example city were analyzed to find out the most popular items. We developed python scripts to collect all the data and to store it in the Google Cloud Platform, more precisely in it's Datastore which is a NoSQL document database. From there it was possible to create a \pg \space instance with the fully-managed relational database service, the so called Google Cloud SQL. For a significant analysis of the raw datasets it was necessary to clean it thoroughly. For this part we developed python scripts as well as different database queries. \newline For the financial part we used a benchmark of Germany's catering industry which includes lots of different key figures like the number of guests per day or the average consumption per guest as well as the revenues and expenses in detail in order to act as economically as possible.