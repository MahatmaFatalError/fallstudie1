\section{Introduction}
\label{sec:introduction}
%Allgemeine Einführung in das Thema
Nowadays lots of people want to realize their dream to start a own business - often its associated with a own restaurant. Therefor a well-conceived concept is absolutely necessary because you have to deal with a lot of different requirements in relation to the financial aspects, the siting and the customers. All these factors should be well-matched to have a unique feature as a restaurant and to be competitive. \newline
%Überleitung zu den Lösungsansätzen
The purpose of this paper is to give recommendations for a new restaurant business in Germany. Hereby the focus is to find an appropriate city for opening a restaurant based on various data analytical observations like restaurant ratings as well as financial conditions and cost structure.
In order to learn which aspects are important for a successful restaurant business the research starts by investigating already existing restaurants. Online community platforms are very useful here. Customers can leave reviews and rate their restaurant visits, and thereby giving recommendations for other users. One of this platforms is Yelp Inc\footnote{\href{https://www.yelp.com/}{www.yelp.com}} with its thousands of restaurants reviews. Yelp offers an \ac{API} for developers which can be accessed via HTTP to retrieve restaurant data. However, to adhere to monetary constraints like in this case maximum investment volume of 750k\euro{} and anticipated monthly revenue of about 40k\euro{} more information is needed. As a consequence we enrich restaurant review data with census data like population density, buying power and also rent prices for commercial buildings. The main idea is to combine all the mentioned facts to calculate a \ac{KPI} to find the most appropriate cities in Germany for starting a restaurant business. In addition, text analysis of reviews provide detailed insights about the existing restaurants and their favorite dishes for different food categories. Finally, to address both financial aspects like rental or personnel costs and also predict the anticipated revenue, best practices of Germany's catering industry are taken into consideration to recommend detailed configuration of a restaurant. In order to ingest, store and analyze the data from various data sources we introduce a scalable data pipeline solution based on the Google Cloud Platform\footnote{\href{https://cloud.google.com/}{cloud.google.com}}.

