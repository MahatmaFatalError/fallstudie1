\section{Discussion}
\label{sec:discussion}
The aim of this paper is to recommend which cities in Germany fit best to open a new restaurant of a certain kind, considering various constraints like financial aspects or availability of gastronomy real estates. To achieve this, we first collected the restaurant data of Yelp into the \ac{GCP} by an scalable data pipeline implementation. Further, we enriched the restaurant data by integrating more data sources like population density, buying power and rent prices. 
Among historical and partially aggregated data such as Yelp we also incorporated real-time real estate data such as ImmobilienScout24. Subsequent, the data got cleansed, harmonized and standardized for analysis. \newline
Starting with inductive statistical methods like multiple regression, the idea to predict revenue as a dependent variable needed to be adapted since no revenue data for Germany's restaurants could be found publicly available. Instead, the user rating has been used as an approximate indicator for a successful restaurant business and thus as a dependent variable. This approach turned out to be unsuitable as the prediction quality of the statistical model suffered from missing attributes. Even the second approach with a decision tree did not provide significantly better results for the analysis. Therefore we developed a \ac{KPI} to rank cities regarding their potential to grow a high-rated and well-attended restaurant. With this \ac{KPI} a top 10 list could be identified. Most of the top 10 cities are located in the federal state of Nordrhein-Westfalen led by Bochum with 7.4\% higher rank compared to the second best city Essen. Afterwards we examined these cities according to categories that were underrepresented in nationwide comparison.
Thus a traditional German restaurant or a café can be considered as a promising restaurant category in Bochum. Another underrepresented category in Bochum is \textit{Thai}. 
This coincides with the analysis of category frequency in Bochum. \textit{Thai} is ranked 17th of altogether 20 categories, counting only restaurants.\footnote{\fullref{app:category}} Thus, \textit{Thai}  category might also be a good choice. Comparing to \fullref{app:items} one notices that 5th and 6th most popular dishes (\textit{fried noodles} occurring 63 and \textit{fried rice} occurring 55 times) in Bochum. On the one hand these are dishes that can be directly connected with Thai cuisine, but it is also noticeable that there are only two Thai restaurants but such a high number of \textit{fried noodles} and \textit{fried rice}. Although there might be other restaurants offering these dishes, only a few of them actually originate from Thai cuisine.%TODO Verstehe die Argumentation nicht. BTW: Stilistischer umschwung von 'we' zu 'one' zu 'you'
As a result, one can recommend that if a \textit{Thai} restaurant is opened, these two dishes should appear within the menu. Additionally, as shown in \fullref{app:fav_items}, one can say that the people in Bochum attach importance to different kinds of desserts. That means if a Thai restaurant should be opened, it can be recommended to compose the dessert menu with great quality and quantity.\\
In order to achieve the required revenue from the use case, a formula was developed using a benchmark of Germany's catering industry. Subsequently, a search was carried out via the ImmobilienScout24 \ac{API} for commercial buildings for sale and rent that checks their affordability.
With the text analysis we examined the individual user reviews from Yelp and determined and aggregated certain words of it. The frequency of the words was used to identify which characteristics customers considered to be most important.\newline
On the basis of various publicly accessible data, we have given a recommendation for investors in which city in Germany it is advisable to start a restaurant business of a certain kind. We have also considered the financial constraints of the use case and offer suggestions for possible commercial buildings.
\\\\
Obviously, there are many different ways to improve the obtained results and to extend this paper. First of all, the analysis of the restaurant data refers exclusively to the Yelp platform. In order to validate this data it would be useful to collect further data from other sources such as Google or Facebook. The same applies to the determination of commercial buildings and the rent index which refer solely to ImmobilienScout24. \newline
Moreover the potential formula could be improved by using further independent variables. In addition to the population data, tourism data of the individual cities in Germany could also be included to obtain more representative results for the ratio between people and restaurants. In terms of the population and the tourists a data acquisition could also be carried out. This could be done through surveys of residents and tourists to receive further and more precise results for the potential formula. For the latter, only positive and independent variables were considered in this paper. However, also the influence of negative factors such as data from crime statistics could be relevant. \newline
When selecting the individual restaurant categories, the influence of ethnic groups could also be significant because of certain neighborhoods where many people with a migration background live. Data from the platform of Google Trends could also be used for a specific restaurant category or food to identify current trends at an early stage. With this it's possible to create or change an appropriate menu. \newline
The review analysis could be enhanced by analyzing the uploaded images on Yelp. With the usage of certain criteria for photo qualities it would be possible to prove and extend the significance of the existing user reviews. The financial analysis could also be improved by using current data of the wages and salaries of Germany‘s population.\newline
