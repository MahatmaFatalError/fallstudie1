\section{Discussion}
\label{sec:discussion}
% 15-20 Zeilen
%
% Vergleich zu bisherigen Algorithmen
% Gab es Ausreißer, die nicht untersucht wurden?
%
% Offene Probleme?
% - Zu wenig daten (im sinne von Breite), zB Tourismus Daten um Besucher zu bestimmen
% - Aufliederung der Städte nach klien, mittel, großstadt
% -
% -
% -
% -

% Notizen
%
% rating als indikator für umsatz
% daten aus verschiedenen datenquellen --> umgesetzt durch architektur
% historische daten & live daten
% trotz limitierte Resosurcen konnten einige Datenquellen gefunden werden um Aussagen über mögliche Städte, Restaurants etc.
%
% Inhalte
% 1. Entwicklung von KPI für Potenzial --> Liste vo Städten
% 2. immoscout api aktuelle live von gastronomie immobilien
%     - literaturrecherche --> Parameter zur Findung Umsatz, Größe, Sitzplatz etc.
% 3. Berechnungen von finanziellen Möglichkeiten durch die Beschränkung in use case description
% 4. Empfehlung für eine Anzahl von Immobilien in einer Stadt mit hohem Potenzial
% 5. Analyse von Reviews --> wichtige qualitative Attribute um ein gutes Rating zu erhalten
% 6. Speisekarten(n) der Stadt --> beliebte Speisen und Services
% 7. Unterrepräsentierte Kategorien könnten genutzt werden um ein Restaurant zu öffnen.
%   Vergleich zu ganz Deutschland kommt die kategorie gut an

\paragraph{Coclusions}
The aim of the paper was to recommend in which cities in Germany a new restaurant of a certain kind can be opened, taking into account the financial framework conditions. To achieve this, we first integrated the data into the Google Cloud Datastore using the Yelp API and specially developed Python code to stored it in a PostgreSQL database instance. Despite the limited resources, we found additional data sources, integrated them and added them to the existing data. The data sources used are live data such as Immoscout or Yelp as well as data collected and published several years ago. After the data had been cleansed and standardized accordingly, the analysis could begin. The original idea was to use sales data as a dependent variable for a regression analysis. To date, however, no turnover data on Germany's restaurant have been publicly available on the Internet. Therefore, we have used the rating as an indicator for a successful restaurant business and as a result, as a dependent variable. However, this approach turned out to be unsuitable as the rating was not significant. This assumption could also be confirmed by an evaluation visualisation Tableau, as there was an even distribution irrespective of the level of the rating. % Verweis auf Grafik 
Even the second approach with a decision tree did not provide significant results for the analysis. Therefore, we have used other parameters to develop our own KPI, with which the potential of possible cities can be assessed. The parameters were on the one hand the restaurant density, which was calculated from the existing restaurants and the population, on the other hand the sum of the reviews and the average rating. These were smoothed with a SoftMax function to eliminate outliers. With this KPI a top 10 list of potential cities could be identified. Within these top 10, most of the cities were in the federal state of Nordrhein- Westfalen, led by Bochum. We then examined the potential cities for underrepresented restaurant categories. For this purpose, the number of restaurants per city was compared with all restaurants in Germany. It was found that for Bochum either a good bourgeois restaurant or a café could be considered as a potential restaurant category.
With the Immoscout API, current live data of commercial real estate from the cities with high potential are used, which can be bought as well as rented.
The financial aspects were examined with the help of a benchmark. With the included parameters like consumption per guest, opening days etc. and the financial requirements from the usecase description a formula was created. This provides an estimate of how many occupied seats a restaurant needs per day. The resulting number of square metres was used to search for possible properties in the top 10 cities. %table cities in chapter?
Using a self-developed algorithm, individual user reviews were picked up and analysed from the Yelp platform. With this review analysis certain words, which refer to the existing restaurants and their food, could be determined and aggregated. By the frequency of the words a statement could be made for offered meals. %result? 
\newline
For the city of Bochum, the menus of existing restaurants were also examined. It turned out that court X and service Y are particularly well received. %result?
The methodn in this paper describes a recommended procedure for finding a restaurant in a city in Germany at a pre-defined financial framework with the help of publicly accessible data.

%future work
\paragraph{Future Work}
Obviously, there are many different ways to improve the obtained results and to extend this thesis. First of all, the analysis of the restaurant data refers exclusively to the Yelp platform. In order to validate this data, it would be useful to collect further data from other sources such as Google, Facebook or Tripadvisor. The same applies to the determination of real estate and the rent index, which refer solely to Immobilienscout. \newline
Moreover the potential formula could be improved by using further independent variables. In addition to the population data, tourism data of the individual cities in Germany could also be included in order to obtain more representative results for the ratio between people and restaurants. Concerning this matter a data acquisition could also be carried out for this thesis. This could be done through surveys of residents and tourists in order to receive further and more precise results in relation to the potential formula. For the latter, only positive, independent variables were considered in this paper. But also the influence of negative factors such as data from crime statistics could be relevant. \newline
When selecting the individual restaurant categories, the influence of ethnic groups could also be significant, as there are certain neighbourhoods, especially in larger cities, where people with a migration background live. Data from the platform of Google Trends could also be used for a specific restaurant category or food to identify current trends at an early stage and to use them as restaurant owner for the menu. \newline
The review analysis could be enhanced by analyzing the uploaded images on Yelp based on certain criteria in order to prove and extend the significance of the existing user reviews. With current data of the wages and salaries of Germany‘s population, another quantitative variable could be added to the financial analysis for an improvement. 
