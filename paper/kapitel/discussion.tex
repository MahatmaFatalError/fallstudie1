\section{Discussion}
\label{sec:discussion}
% 15-20 Zeilen
%
% Vergleich zu bisherigen Algorithmen
% Gab es Ausreißer, die nicht untersucht wurden?
%
% Offene Probleme?
% - Zu wenig daten (im sinne von Breite), zB Tourismus Daten um Besucher zu bestimmen
% - Aufliederung der Städte nach klien, mittel, großstadt
% -
% -
% -
% -

% Notizen
%
% rating als indikator für umsatz
% daten aus verschiedenen datenquellen --> umgesetzt durch architektur
% historische daten & live daten
% trotz limitierte Resosurcen konnten einige Datenquellen gefunden werden um Aussagen über mögliche Städte, Restaurants etc.
%
% Inhalte
% 1. Entwicklung von KPI für Potenzial --> Liste vo Städten
% 2. immoscout api aktuelle live von gastronomie immobilien
%     - literaturrecherche --> Parameter zur Findung Umsatz, Größe, Sitzplatz etc.
% 3. Berechnungen von finanziellen Möglichkeiten durch die Beschränkung in use case description
% 4. Empfehlung für eine Anzahl von Immobilien in einer Stadt mit hohem Potenzial
% 5. Analyse von Reviews --> wichtige qualitative Attribute um ein gutes Rating zu erhalten
% 6. Speisekarten(n) der Stadt --> beliebte Speisen und Services
% 7. Unterrepräsentierte Kategorien könnten genutzt werden um ein Restaurant zu öffnen.
%   Vergleich zu ganz Deutschland kommt die kategorie gut an

This paper describes the possibility to find a location for a restaurant investment with the help of data. 
In a first step, Yelp's data was integrated and analyzed in order to achieve this. It was assumed that Yelp's rating could be used as a dependent variable and as a unit of measurement for sales. In addition, the reviews were used as an important criterion for a qualitative assessment.The categories from Yelp were used as a quantitative expression to find potentially underrepresented restaurant categories. \newline
Many variables were enriched from different data sources to develop key performance indicators to evaluate and compare the potential of cities. \newline 
With the help of population data, for example, a restaurant density was calculated that could be included in the potential analysis.
Data of buying power from German population were evaluated in order to recommend the offer price.
The choice between renting and buying was included in the evaluation in order to give the investors more room to make their own decisions. \newline 
The data were correlated, a regression was carried out and variance analyses were made to reveal a possible significance. 
The application was tested with the help of a use case which required both a financial framework for the initial investment and a minimum sale.  \newline
With the help of a benchmark, the size of the restaurant and the possible seat distribution could be calculated in order to compare this with live data from the real estate platforms in relation to the use case investment volume.
This resulted in a top 10 city list as a recommendation for an investment. 
To find the type and popularity of a favorite meal, a sentiment analysis was made of various menus. In addition to the offer price, this also resulted in a recommendation of the most popular dishes and services. 

%
% Zwischenresumee
% WAS, WO, WIE teuer
% mehr Datenquellen um die vorhandenen Datenquellen zu valideiren wie Google, Facebook andere MietKaufportale immowelt etc.
% bezug zur Intrdocution. Was war die Aufgabenstellung --> Alle Anforderungen konnte datengetrieben beantwortet werden.
% resultierende Ergebnisse konnten nachvollziehbar dargestelt werden.
% qualitative Informationen fehlen
% touristen
% wie viele menschen waren tasächlich in einem restaurant?
%
% Future Work
There are many different ways to improve the obtained results and to extend this thesis. First of all, the analysis of the restaurant data refers exclusively to the Yelp platform. In order to validate this data, it would be useful to collect data from data sources such as Google, Facebook or Tripadvisor. The same applies to the determination of real estate and the rent index, which refer solely to Immobilienscout. \newline
Moreover the potential formula could be improved by using further independent variables. In addition to the population data, tourism data of the individual cities in Germany could also be included in order to obtain more representative results for the ratio between people and restaurants. Concerning this matter a data acquisition could also be carried out for this thesis. This could be done through surveys of residents and tourists in order to receive further and more precise results in relation to the potential formula. For the latter, only positive, independent variables were considered in this paper. But also the influence of negative factors such as data from crime statistics could be relevant. \newline
When selecting the individual restaurant categories, the influence of ethnic groups could also be significant, as there are certain neighbourhoods, especially in larger cities, where people with a migration background live. Data from the platform of Google Trends could also be used for a specific restaurant category or food to identify current trends at an early stage and to use them as restaurant owner for the menu. \newline
The review analysis could be enhanced by analyzing the uploaded images on Yelp based on certain criteria in order to prove and extend the significance of the existing user reviews. \newline
With current data of the wages and salaries of Germany‘s population, another quantitative variable could be added to the financial analysis for an improvement. 