\section{Discussion}
\label{sec:discussion}
% 15-20 Zeilen
%
% Vergleich zu bisherigen Algorithmen
% Gab es Ausreißer, die nicht untersucht wurden?
%
% Offene Probleme?
% - Zu wenig daten (im sinne von Breite), zB Tourismus Daten um Besucher zu bestimmen
% - Aufliederung der Städte nach klien, mittel, großstadt
% -
% -
% -
% -

% Notizen
%
% rating als indikator für umsatz
% daten aus verschiedenen datenquellen --> umgesetzt durch architektur
% qualitative Informationen fehlen
% touristen
% wie viele menschen waren tasächlich in einem restaurant?
% historische daten & live daten
% trotz limitierte Resosurcen konnten einige Datenquellen gefunden werden um Aussagen über mögliche Städte, Restaurants etc.
%
% Inhalte
% 1. Entwicklung von KPI für Potenzial --> Liste vo Städten
% 2. immoscout api aktuelle live von gastronomie immobilien
%     - literaturrecherche --> Parameter zur Findung Umsatz, Größe, Sitzplatz etc.
% 3. Berechnungen von finanziellen Möglichkeiten durch die Beschränkung in use case description
% 4. Empfehlung für eine Anzahl von Immobilien in einer Stadt mit hohem Potenzial
% 5. Analyse von Reviews --> wichtige qualitative Attribute um ein gutes Rating zu erhalten
% 6. Speisekarten(n) der Stadt --> beliebte Speisen und Services
% 7. Unterrepräsentierte Kategorien könnten genutzt werden um ein Restaurant zu öffnen.
%   Vergleich zu ganz Deutschland kommt die kategorie gut an
%
% Zwischenresumee
% WAS, WO, WIE teuer
% mehr Datenquellen um die vorhandenen Datenquellen zu valideiren wie Google, Facebook andere MietKaufportale immowelt etc.
% bezug zur Intrdocution. Was war die Aufgabenstellung --> Alle Anforderungen konnte datengetrieben beantwortet werden.
% resultierende Ergebnisse konnten nachvollziehbar dargestelt werden.
%
% Future Work
% negative faktoren zu potenzailanalyse dazunehmen aus zb kriminalstatistik
% daten ereheben um bessere, genauere Ergebnisse zu liefern:
% Bilder von Reviews analysieren, aktuelle Lohndaten für die verbesserte Finanzanalyse, Touristik informationen, Trendanalyse miteinbeziehen
% Umfargen von Einwohnern um die Top Cities/Potenzial zu analysieren
% haben ehtnische gruppen einen einfluss auf die auswahl der "kategorie"
