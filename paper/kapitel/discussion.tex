\section{Discussion}
\label{sec:discussion}
The aim of the paper was to recommend in which cities in Germany a new restaurant of a certain kind can be opened and to consider the financial constraints. To achieve this, we first collected the restaurant data of Yelp into the \ac{GCP}. Further we enriched more data sources like population density, buying power and rent prices and integrated them to the existing data. The used data sources are real-time data such as ImmobilienScout24 as well as historical data such as Yelp. After that the data was cleansed and standardized for analysis. The original idea was to use the revenue as a dependent variable for a regression analysis. However, no revenue data for Germany's restaurants could be found publicly available on the Internet. Instead, we have used the rating as an indicator for a successful restaurant business and as a dependent variable. This approach turned out to be unsuitable as the regression model and its variables were not reliable. Even the second approach with a decision tree did not provide significant results for the analysis. Therefore we have used other parameters to develop a \ac{KPI} to obtain the potential of possible cities. With this \ac{KPI} a top 10 list could be identified. Within these top 10 most of the cities are in the federal state of Nordrhein-Westfalen led by Bochum. Afterwards we analyzed the potential cities for underrepresented restaurant categories. For this purpose, the number of restaurants per city was compared with all restaurants in Germany. 
It could be determined that a traditional German restaurant or a café could be considered as a possible restaurant category in Bochum. Another underrepresented category from \fullref{subsubsec:potential} is \textit{Thai}. This coincides with the analysis of the categories in Bochum, where \textit{Thai} is represented on place 17 of altogether 20 categories and only with two restaurants.\footnote{\fullref{app:category}} Thus this category would also be at least a good choice. If one compares this now with \fullref{app:items} one notices that on place 5 and 6 the dishes \textit{fried noodles} and \textit{fried rice} stand with respectively 63 and 55 occurrences. On the one hand these are dishes that can be directly connected with Thai cuisine, but it is also noticeable that there are only two Thai restaurants but such a high number of \textit{fried noodles} and \textit{fried rice}.
This discovery can be interpreted that although there are restaurants offering these dishes, only a few of them actually originate from Thai cuisine. As a result, one can recommend that if a Thai restaurant is opened, these two dishes should appear within the menu.
As you can see from \fullref{app:fav_items} the favorite dish in Bochum is the \textit{rumpsteak} with a total count of 7. On place two is the \textit{mixed ice cream} with 4 occurrences. From these Top 10 one can read that the people in Bochum attach great importance to different kinds of desserts. That means if a Thai restaurant should be opened, it can be recommended to offer a larger dessert menu, because desserts are especially popular in Bochum.
In order to achieve the required revenue from the use case, a formula was developed using a benchmark of Germany's catering industry. Subsequently, a search was carried out via the ImmobilienScout24 \ac{API} for commercial buildings for sale and rent that checks their affordability.
With the text analysis we examined the individual user reviews from Yelp and determined and aggregated certain words of it. The frequency of the words was used to identify which characteristics customers considered to be most important.
As the menu can be also an important indicator of success we analyzed the menus of restaurants in Bochum. It turned out that rumpsteak is the most popular item of customers. \newline
On the basis of various publicly accessible data, we have given a recommendation for investors in which city in Germany it is advisable to start a restaurant business of a certain kind. We have also considered the financial constraints of the use case and offer suggestions for possible commercial buildings.

%future work
Obviously, there are many different ways to improve the obtained results and to extend this paper. First of all, the analysis of the restaurant data refers exclusively to the Yelp platform. In order to validate this data it would be useful to collect further data from other sources such as Google or Facebook. The same applies to the determination of commercial buildings and the rent index which refer solely to ImmobilienScout24. \newline
Moreover the potential formula could be improved by using further independent variables. In addition to the population data, tourism data of the individual cities in Germany could also be included to obtain more representative results for the ratio between people and restaurants. In terms of the population and the tourists a data acquisition could also be carried out. This could be done through surveys of residents and tourists to receive further and more precise results for the potential formula. For the latter, only positive and independent variables were considered in this paper. However, also the influence of negative factors such as data from crime statistics could be relevant. \newline
When selecting the individual restaurant categories, the influence of ethnic groups could also be significant because of certain neighborhoods where many people with a migration background live. Data from the platform of Google Trends could also be used for a specific restaurant category or food to identify current trends at an early stage. With this it's possible to create or change an appropriate menu. \newline
The review analysis could be enhanced by analyzing the uploaded images on Yelp. With the usage of certain criteria for photo qualities it would be possible to prove and extend the significance of the existing user reviews. The financial analysis could also be improved by using current data of the wages and salaries of Germany‘s population.




%Another information from \fullref{subsubsec:potential} is that one of the underrepresented categories is \textit{Thai}.
%This coincides with the analysis of the categories in Bochum, where \textit{Thai} is represented on place 17 of altogether 20 categories and only with two restaurants.\footnote{\fullref{app:category}}
%Thus this category would be at least a first selection.